\begin{enumerate}[label=\thesubsection.\arabic*.,ref=\thesubsection.\theenumi]

\item  $D$ is a point on the side $BC$ of a $\triangle ABC$ such that  $\angle  ADC =  \angle  BAC$. Show that 
\begin{align}
	\label{eq:tri-sim}	
	CA^2 = CB.CD
\end{align}
%
	\\
		\solution
	See \figref{fig:tri-sim}.	
\begin{align}
	\frac{x}{\sin\brak{A+C}} &= 
	\frac{b}{\sin A}  
\quad 
		\brak{\triangle ADC},
		\\
	\implies	\frac{x}{\sin B} &= 
	\frac{b}{\sin A}  
	\\
\implies	\frac{x}{b} = 
	\frac{\sin B}{\sin A}  &= 
	\frac{b}{a}  \quad \brak{\text{ sine formula }}
\end{align}
yielding
	\eqref{eq:tri-sim}.	
\begin{figure}[H]
	\begin{center}
			\resizebox{0.6\columnwidth}{!}{\input{figs/tri/tri-sim.tex}}
	\end{center}
	\caption{}
	\label{fig:tri-sim}	
\end{figure}
\item   $D$ is a point on side $BC$ of  $\triangle  ABC$ such that
$\frac{BD}{CD}= \frac{AB}{AC}  $.  Prove that $AD$ is the bisector of  $\angle  BAC$.
\\
\solution 
	See \figref{fig:tri-ang-bis}.	
\begin{align}
	\frac{x}{a-x} &= 
	\frac{c}{b}  \quad \brak{\text{ given }}
	\\
	\frac{c}{\sin\phi} &= 
	\frac{x}{\sin \theta}  \quad \brak{\triangle ABD}
	\\
	\frac{a-x}{\sin\brak{A-\theta}} &= 
	\frac{b}{\sin 180-\phi}  \quad \brak{\triangle ACD}
	\\
	&=\frac{b}{\sin \phi}
\end{align}
using the sine formula.  Multiplying all the above equations yields
\begin{align}
	\sin \brak{A-\theta}
	=\sin \theta \implies \theta = \frac{A}{2}
\end{align}
\begin{figure}[H]
	\begin{center}
			\resizebox{0.6\columnwidth}{!}{\input{figs/tri/tri-ang-bis.tex}}
	\end{center}
	\caption{}
	\label{fig:tri-ang-bis}	
\end{figure}
%
\item  $ABC$ is a triangle in which  $\angle  ABC > 90\degree$ and $AD  \perp  CB$ produced. Prove that
\begin{align}
	\label{eq:tri-obtuse}
 AC^2= AB^2 + BC^2 + 2 BC . BD.
\end{align}
\solution
	See \figref{fig:tri-obtuse}.	
\begin{align}
	\label{eq:tri-obtuse-1}
	\cos B &= \frac{x}{c} \quad \brak{\triangle ADB}
	\\
	b^2 &= a^2 + c^2 -2ac \cos \brak{180-B} \quad \brak{\triangle ABC}
	\\
	&= a^2 + c^2 +2ac \cos B 
	\label{eq:tri-obtuse-2}
\end{align}
using the cosine formula.
Substituting from 
	\eqref{eq:tri-obtuse-1}
	in
	\eqref{eq:tri-obtuse-2}
	yields 
	\eqref{eq:tri-obtuse}.
\begin{figure}[H]
	\begin{center}
			\resizebox{0.6\columnwidth}{!}{\input{figs/tri/tri-obtuse.tex}}
	\end{center}
	\caption{}
	\label{fig:tri-obtuse}	
\end{figure}
\item In a right triangle, prove that the line-segment joining the mid-point of the hypotenuse to the opposite vertex is half the hypotenuse.
	\\
	\solution
	In \figref{fig:tri-hyp}	
\begin{align}
	\label{eq:tri-hyp}	
	\frac{x}{\sin C} &= \frac{b/2}{\sin \theta} \quad \brak{\triangle BDC}
	\\
	\frac{x}{\sin A} &= \frac{b/2}{\sin \brak{90-\theta}} \quad \brak{\triangle BDA}
	\\
\implies	\frac{x}{\cos C} &= \frac{b/2}{\cos \theta} 
	\label{eq:tri-hyp-1}	
\end{align}
From 
	\eqref{eq:tri-hyp}	
	and
	\eqref{eq:tri-hyp-1},
\begin{align}
	\brak{\frac{\sin C}{x}}^2
	+
	\brak{\frac{\cos C}{x}}^2
	&= 
	\brak{\frac{\cos \theta}{\frac{b}{2}}}^2
	+
	\brak{\frac{\sin \theta}{\frac{b}{2}}}^2
	\\
	\implies x &= \frac{b}{2} 
\end{align}
using \eqref{eq:tri_sin_cos_id}.
\begin{figure}[H]
	\begin{center}
			\resizebox{0.6\columnwidth}{!}{\input{figs/tri/tri-hyp.tex}}
	\end{center}
	\caption{}
	\label{fig:tri-hyp}	
\end{figure}
\item $ABCD$ is a trapezium in which $AB  \parallel  DC$ and its diagonals intersect each other at the point $O$. Show
that
\begin{align}
	\label{eq:tri-trap}	
\frac{AO}{ BO}=\frac{CO}{  DO}
\end{align}
\begin{figure}[H]
	\begin{center}
			\resizebox{0.6\columnwidth}{!}{\input{figs/tri/tri-trap.tex}}
	\end{center}
	\caption{}
	\label{fig:tri-trap}	
\end{figure}
\solution 
	In \figref{fig:tri-trap}, $\because AB \parallel CD$	
\begin{align}
	\frac{AO}{\sin \phi} &= \frac{BO}{\sin \theta} \quad \brak{\triangle OAB}
	\\
	\frac{CO}{\sin \phi} &= \frac{DO}{\sin \theta} \quad \brak{\triangle ODC}
\end{align}
yielding
	\eqref{eq:tri-trap}	
	after simplification.
\item $O$ is any point inside a rectangle $ABCD$. Prove that 
\begin{align}
	OB^2+OD^2 = OA^2+OC^2
	\label{eq:tri-rect}	
\end{align}
	\solution
	In 
	\figref{fig:tri-rect},	
from \eqref{ch1_budh_basic}
\begin{align}
	p \cos \theta_1+q \sin \theta_2 = a \quad \brak{\triangle OAB}
	\\
	r \cos \theta_3+s \sin \theta_4 = a \quad \brak{\triangle OAB}
	\\
	p \cos \theta_1+s \sin \theta_4 = b \quad \brak{\triangle OAB}
	\\
	r \cos \theta_3+q \sin \theta_2 = b \quad \brak{\triangle OAB}
\end{align}
Subtracting the first two and second two equations respectively,
\begin{align}
	p \cos \theta_1 
	-s \sin \theta_4  
= r \cos \theta_3-q \sin \theta_2
\\
	p \cos \theta_1+s \sin \theta_4 = 
	r \cos \theta_3+q \sin \theta_2  
\end{align}
Squaring and adding and using 
\eqref{eq:tri_sin_cos_id}
yields
	\eqref{eq:tri-rect}.	
\begin{figure}[H]
	\begin{center}
			\resizebox{0.6\columnwidth}{!}{\input{figs/tri/tri-rect.tex}}
	\end{center}
	\caption{}
	\label{fig:tri-rect}	
\end{figure}
\item  In  $\triangle  ABC, AB = 6\sqrt{3} cm, AC = 12 cm$ and $BC = 6 cm$. Find the angle $B$.
	\\
	\solution Using 
\eqref{eq:tri_cos_form},
\begin{align}
	\cos B &= 
 \frac{c^2+a^2-b^2}{2ca}=0
	\\
	\implies B &= 90\degree
\end{align}
\end{enumerate}
