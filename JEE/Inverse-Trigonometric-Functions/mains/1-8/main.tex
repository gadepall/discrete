\iffalse
\documentclass[journal]{IEEEtran}
\usepackage[a5paper, margin=10mm]{geometry}
%\usepackage{lmodern} % Ensure lmodern is loaded for pdflatex
\usepackage{tfrupee} % Include tfrupee package

\let\negmedspace\undefined
\let\negthickspace\undefined
\usepackage{gvv-book}
\usepackage{gvv}
\usepackage{cite}
\usepackage{amsmath,amssymb,amsfonts,amsthm}
\usepackage{algorithmic}
\usepackage{graphicx}
\usepackage{textcomp}
\usepackage{xcolor}
\usepackage{txfonts}
\usepackage{listings}
\usepackage{enumitem}
\usepackage{mathtools}
\usepackage{gensymb}
\usepackage{comment}
\usepackage[breaklinks=true]{hyperref}
\usepackage{tkz-euclide} 
\usepackage{listings}                                        
%\def\inputGnumericTable{}                                 
\usepackage[latin1]{inputenc}                                
\usepackage{color}                                            
\usepackage{array}                                            
\usepackage{longtable}                                       
\usepackage{calc}                                             
\usepackage{multirow}                                         
\usepackage{hhline}                                           
\usepackage{ifthen}                                           
\usepackage{lscape}
\usepackage{tabularx}
\usepackage{array}
\usepackage{float}
\usepackage{multicol}

\newcommand{\BEQA}{\begin{eqnarray}}
\newcommand{\EEQA}{\end{eqnarray}}
%\newcommand{\define}{\stackrel{\triangle}{=}}

\setlength{\headheight}{1cm} % Set the height of the header box
\setlength{\headsep}{0mm}     % Set the distance between the header box and the top of the text


%\usepackage[a5paper, top=10mm, bottom=10mm, left=10mm, right=10mm]{geometry}


\setlength{\intextsep}{10pt} % Space between text and floats

% Marks the beginning of the document
\begin{document}
\onecolumn
\bibliographystyle{IEEEtran}
\vspace{3cm}

%\renewcommand{\theequation}{\theenumi}
\numberwithin{equation}{enumi}
\numberwithin{figure}{enumi}
\renewcommand{\thefigure}{\theenumi}
\renewcommand{\thetable}{\theenumi}

\title{Mains - 14.A+B}
\author{Shiven Bajpai}
\section{mains}
\maketitle
\fi

%\begin{enumerate}
	\item{
			$\cos^{-1}\brak{\sqrt{\cos\alpha}}-\tan^{-1}\brak{\sqrt{\cos\alpha}}$ , then $\sin x =$ \hfill (2002)
		\begin{multicols}{4}
		\begin{enumerate}
			\item{$\tan^2 \brak{\frac{\alpha}{2}}$}
			\columnbreak
			\item{$\cot^2 \brak{\frac{\alpha}{2}}$}
			\columnbreak
			\item{$\tan\alpha$}
			\columnbreak
			\item{$\cot \brak{\frac{\alpha}{2}}$}
		\end{enumerate}
		\end{multicols}
	}
	\item{
			The trignometric equation $\sin^{-1} x = 2 \sin^{-1}a$ has a solution for \hfill (2003)
		\begin{multicols}{2}
		\begin{enumerate}
			\item{$\abs{\alpha}\geq\frac{1}{\sqrt{2}}$}
			\item{$\frac{1}{2} < \abs{\alpha} < \frac{1}{\sqrt{2}}$}
			\columnbreak
			\item{all real values of $a$}
			\item{$\abs{\alpha} < \frac{1}{2}$}
		\end{enumerate}
		\end{multicols}
	}
	\item{
			If $\cos^{-1}x - \cos^{-1}\frac{y}{2} = \alpha$, then $4x^2 - 4xy \cos \alpha + y^2$ is equal to \hfill (2005)
		\begin{multicols}{4}
		\begin{enumerate}
			\item{$2 \sin 2\alpha$}
			\columnbreak
			\item{$4$}
			\columnbreak
			\item{$4 \sin^2 \alpha$}
			\columnbreak
			\item{$-4 \sin^2 \alpha$}
		\end{enumerate}
		\end{multicols}
	}
	\item{
			If $\sin^{-1} \brak{\frac{x}{5}} + \cosec^{-1}\brak{\frac{5}{4}} = \frac{\pi}{2}$, then the value of $x$ is \hfill (2007)
		\begin{multicols}{4}
		\begin{enumerate}
			\item{$4$}
			\columnbreak
			\item{$5$}
			\columnbreak
			\item{$1$}
			\columnbreak
			\item{$3$}
		\end{enumerate}
		\end{multicols}
	}
	\item{
			The value of $\cot\brak{\cosec^{-1}\frac{5}{3} + \tan^{-1}\frac{2}{3}}$
		\begin{multicols}{4}
		\begin{enumerate}
			\item{$\frac{6}{17}$}
			\columnbreak
			\item{$\frac{3}{17}$}
			\columnbreak
			\item{$ \frac{4}{17}$}
			\columnbreak
			\item{$ \frac{5}{17}$}
		\end{enumerate}
		\end{multicols}
	}
	\item{
			If $x,y,z$ are in AP and $\tan^{-1}x, \tan^{-1}y \text{ and} \tan^{-1}z$ are also in A.P, then \hfill (JEE M 2013)
		\begin{multicols}{4}
		\begin{enumerate}
			\item{$x=y=z$}
			\columnbreak
			\item{$2x=3y=6z$}
			\columnbreak
			\item{$6x=3y=2z$}
			\columnbreak
			\item{$6x=4y=3z$}
		\end{enumerate}
		\end{multicols}
	}
	\item{
			Let $\tan^{-1}y = \tan^{-1}x + \tan^{-1}\brak{\frac{2x}{1-x^2}}$, where $\abs{x} < \frac{1}{\sqrt{3}}$. Then a value of $y$ is \hfill (JEE M 2015)
		\begin{multicols}{4}
		\begin{enumerate}
			\item{$\frac{3x - x^3}{1 + 3x}$}
			\columnbreak
			\item{$\frac{3x + x^3}{1 + 3x}$}
			\columnbreak
			\item{$\frac{3x - x^3}{1 - 3x}$}
			\columnbreak
			\item{$\frac{3x + x^3}{1 - 3x}$}
		\end{enumerate}
		\end{multicols}
	}
	\item{
			If $\cos^{-1}\brak{\frac{2}{3x}} + \cos^{-1}\brak{\frac{3}{4x}} = \frac{\pi}{2} \brak{x > \frac{3}{4}}$, then $x$ is equal to \hfill (JEE M 2019 - 9 Jan M)
		\begin{multicols}{4}
		\begin{enumerate}
			\item{$\frac{\sqrt{145}}{12}$}
			\columnbreak
			\item{$\frac{\sqrt{145}}{10}$}
			\columnbreak
			\item{$\frac{\sqrt{146}}{12}$}
			\columnbreak
			\item{$\frac{\sqrt{145}}{11}$}
		\end{enumerate}
		\end{multicols}
	}
%\end{enumerate}

%\end{document}
