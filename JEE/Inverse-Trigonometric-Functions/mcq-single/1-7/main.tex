\iffalse
\title{Assignment}
\author{Arjun Pavanje}
\section{mcq-single}
\fi
\item The value of $\tan \sbrak{ \cos ^{-1}\brak{\frac{4}{5}}+\tan^{-1}\brak{\frac{2}{3}} }$ is
\hfill \brak{1983-1Mark}
\begin{enumerate}
\begin{multicols}{2}
\item $\frac{6}{17}$
\columnbreak
\item$\frac{7}{16}$
\end{multicols}
\begin{multicols}{2}
\item $\frac{16}{7}$ 
\columnbreak
\item None
\end{multicols}
\end{enumerate}
\item If we consider only the principle values of the inverse trigonometric functions then the value of
\begin{align*}
\tan \brak{\cos ^{-1}\brak{\frac{1}{5\sqrt{2}}}-\sin ^{-1}\brak{\frac{4}{\sqrt{17}}}}
\end{align*}
is
\hfill \brak{1994}
\begin{enumerate}
\begin{multicols}{2}
\item $\frac{\sqrt{29}}{3}$ 
\columnbreak
\item $\frac{29}{3}$
\end{multicols}
\begin{multicols}{2}
\item $\frac{\sqrt{3}}{29}$ 
\columnbreak
\item $\frac{3}{29}$ 
\end{multicols}
\end{enumerate}
\item The number of real solutions of
\begin{align*}
\tan ^{-1}\brak{\sqrt{x\brak{x-1}}}+\sin ^{-1}\brak{\sqrt{x^2+x+1}}=\frac{\pi}{2}
\end{align*}
is 
\hfill \brak{1999-2Marks}
\begin{enumerate}
\begin{multicols}{2}
\item zero 
\columnbreak
\item one 
\end{multicols}
\begin{multicols}{2}
\item two 
\columnbreak
\item infinite
\end{multicols}
\end{enumerate}
\item If
\begin{align*}
\sin ^{-1}\brak{x-\frac{x^2}{2}+\frac{x^3}{4}-\ldots}+ \cos ^{-1}\brak{x^2-\frac{x^4}{2}+\frac{x^6}{4}-\ldots}=\frac{\pi}{2}
\end{align*}
for $0<\abs{x}<\sqrt{2}$, then $x$ equals 
\hfill \brak{2001S}
\begin{enumerate}
\begin{multicols}{2}
\item $\frac{1}{2}$
\columnbreak
\item $1$ 
\end{multicols}
\begin{multicols}{2}
\item $-\frac{1}{2}$ 
\columnbreak
\item $-1$
\end{multicols}
\end{enumerate}
\item The value of $x$ for which 
\begin{align*}
\sin\brak{\cot^{-1}\brak{1+x}}=\cos \brak{\tan ^{-1}\brak{x}}
\end{align*}
is 
\hfill \brak{2004S}
\begin{enumerate}
\begin{multicols}{2}
\item $\frac{1}{2}$ 
\columnbreak
\item $1$
\end{multicols}
\begin{multicols}{2}
\item $0$ 
\columnbreak
\item $-\frac{1}{2}$
\end{multicols}
\end{enumerate}
\item  If $0<x<1$, then 
\begin{multline*}
\sqrt{1+x^2}\sbrak{\cbrak{ x\cos \brak{\cot ^{-1}\brak{x}}+ \sin \brak{\cot ^{-1}\brak{x}}}^2 - 1}^{\frac{1}{2}}
\end{multline*}
is
\hfill \brak{2008}
\begin{enumerate}
\begin{multicols}{2}
\item $\frac{x}{\sqrt{1+x^2}}$ 
\columnbreak
\item $x$
\end{multicols}
\begin{multicols}{2}
\item $x\sqrt{1+x^2}$ 
\columnbreak
\item $\sqrt{1+x^2}$
\end{multicols}
\end{enumerate}
\item The value of 
\begin{align*}
\cot \brak{\sum_{n=1}^{23} \cot ^{-1}\brak{1+\sum_{k=1}^{n} 2k}}
\end{align*}
is
\hfill \brak{JEE Adv.2013}
\begin{enumerate}
\begin{multicols}{2}
\item $\frac{23}{25}$ 
\columnbreak
\item $\frac{25}{23}$ 
\end{multicols}
\begin{multicols}{2}
\item $\frac{23}{24}$ 
\columnbreak
\item $\frac{24}{23}$
\end{multicols}
\end{enumerate}
