\begin{enumerate}[label=\thesubsection.\arabic*,ref=\thesubsection.\theenumi]
    \item A polygon of nine sides, each of length $2$, is inscribed in a circle. The radius of the circle is \rule{1cm}{0.1pt}. \hfill (1987) 
    \item A circle is inscribed in a equilateral triangle of a side $a$. The area of any square inscribed in this circle is \rule{1cm}{0.1pt}. \hfill (1994)  
    \item In a triangle $ABC$, $a:b:c = 4:5:6$. The ratio of the radius of the circumstances to that of the incircle is \rule{1cm}{0.1pt}. \hfill (1996) 
        \item The sum of the radii of inscribed and circumscribed circles for an $n$ sided regular polygon of side $a$, is \hfill\brak{2003}
\begin{multicols}{4}
\begin{enumerate}
        \item $\frac{a}{4} \cot{\brak{\frac{\pi}{2n}}}$         
        \item $ a \cot{\brak{\frac{\pi}{n}}}$ 
        \item $\frac{a}{2} \cot{\brak{\frac{\pi}{2n}}}$ 
        \item $ a \cot{\brak{\frac{\pi}{2n}}}$
\end{enumerate}
\end{multicols}
\item For a regular polygon, let $r$ and $R$ be the radii of the inscribed and the circumscribed circles. A false statement among the following is \hfill\brak{2010}
\begin{enumerate}
        \item There is a regular polygon with $\frac{r}{R}=\frac{1}{\sqrt{2}}$                    
        \item There is a regular polygon with $\frac{r}{R}=\frac{2}{3}$ 
        \item There is a regular polygon with $\frac{r}{R}=\frac{\sqrt{3}}{2}$ 
        \item There is a regular polygon with $\frac{r}{R}=\frac{1}{2}$
\end{enumerate}
    \item Let $A_{0}A_{1}A_{2}A_{3}A_{4}A_{5}$ be a regular hexagon inscribed in a circle of unit radius. Then the product of the lengths of the line segments $A_{0}A_{1}$,$A_{0}A_{2}$ and $A_{0}A_{4}$ is 
    \hfill{(1998)}
    \begin{multicols}{4}
    	\begin{enumerate}
    		\item ${\frac{3}{4}}$
    		\item $3\sqrt{3}$
    		\item $3$
    		\item ${\frac{3\sqrt{3}}{2}}$
    	\end{enumerate}
    \end{multicols}
    \item In a triangle $PQR$, ${P}$ is the largest angle and $\cos{P} = \frac{1}{3}$. Further the incircle of the triangle touches the sides $PQ,QR \text{ and } RP$ at ${N},{L} \text{ and } {M}$ respectively, such that the lengths of $PN, QL \text{ and } RM$ are consecutive even integers. Then possible length(s) of the side(s) of the triangle is (are)
    \hfill{(2013)}
    \begin{multicols}{4}
    	\begin{enumerate}
    		\item $16$
    		\item $24$
    		\item $18$
    		\item $22$
    	\end{enumerate}
    \end{multicols}
    \item In a triangle $XYZ$, let $x,y,z$ be the lengths of sides opposite to angles ${X},{Y},{Z} \text{ and } 2s = x+y+z$. If $${\frac{s-x}{4}}={\frac{s-y}{3}}={\frac{s-z}{2}}$$ and area of the incircle of the triangle $XYZ$ is ${\frac{8\pi}{3}},$
    \hfill{(2016)}
    \begin{enumerate}
    	\item area of the triangle is $6\sqrt{6}$
    	\item the radius of circumcirle of $XYZ$ is ${\frac{35\sqrt{6}}{6}}$
    	\item $\sin\frac{X}{2}\sin\frac{Y}{2}\sin\frac{Z}{2} = \frac{4}{35}$
    	\item $\sin^{2}\brak{\frac{X+Y}{2}}$ = $\frac{3}{5}$
    \end{enumerate}
    \item In a triangle $PQR$, let $\angle PQR = 30\degree$ and the sides $PQ \text{ and } QR$ have lengths $10\sqrt{3}$ and $10$ respectively. Then which of the following statements is (are)  TRUE?
    \hfill{(2018)}
    \begin{enumerate}
    	\item $\angle QPR = 45\degree$
    	\item the area of the triangle $PQR$ is $25\sqrt{3}$ and $\angle QRP = 120\degree$
    	\item the radius of the incircle of triangle $PQR$ is $10\sqrt{3}-15$
    	\item the radius of circumcirle $PQR$ is $100\pi$
    \end{enumerate}
    \item In a non-right-angle triangle $\triangle PQR$, let $p,q,r$ denote the lengths of the sides opposite to the angles at ${P},{Q},{R}$ respectively. The median from ${R}$ meets the side $PQ$ at ${S}$, the perpendicular from ${P}$ meets the side $QR$ at ${E}$, $RS \text{ and } PE$ intersect at ${O}$. If $p = \sqrt{3}$, $q = 1$ and the radius of the circumcircle at $\triangle PQR$ equals $1$, then which of the following options is (are) correct.
    \hfill{(2018)}
    \begin{enumerate}
    	\item Radius of incircle of $\triangle PQR$ = $\frac{\sqrt{3}}{2}\brak{2-\sqrt{3}}$
    	\item Area of $\triangle SOE = \frac{\sqrt{3}}{12}$
    	\item Length of $OE = \frac{1}{6}$
    	\item Length of $RS = \frac{\sqrt{7}}{2}$
    \end{enumerate}
\item Which of the following pieces of data does NOT uniquely determine an acute-angled triangle $\triangle ABC$ (${R}$ being the radius of the circumcircle)?
\hfill (2002)
		\begin{multicols}{4}
\begin{enumerate}
\item $a, \sin A, \sin B$
\item $a, b, c$
\item $a, \sin B, R$
\item $a, \sin A, R$
\end{enumerate}
    \end{multicols}
\item One angle of an isosceles $\triangle$ is $120\degree$ and radius of its incircle $= \sqrt{3}$. Then the area of the triangle in sq. units is 
\hfill (2006)
		\begin{multicols}{4}
\begin{enumerate}
\item $7+12\sqrt{3}$
\item $12-7\sqrt{3}$
\item $12+7\sqrt{3}$
\item $4\pi$
\end{enumerate}
    \end{multicols}
%
\item Let $ABCD$ be a quadrilateral with area $18$, with side $AB$ parallel to the side $CD$ and $2AB = CD$. Let $AD$ be perpendicular to $AB$ and $CD$. If a circle is drawn inside the quadrilateral $ABCD$ touching all the sides, then the radius is
\hfill (2007)
		\begin{multicols}{4}
\begin{enumerate}
\item $3$
\item $2$
\item $\frac{3}{2}$
\item $1$
\end{enumerate}
\end{multicols}
%
     	\item If a triangle is inscribed in a circle, then the product of any two sides of the triangle is equal to the product of the diameter and perpendicular distance of the third side from the opposite vertex. Prove the above statement.
     \hfill {(1979)}
     	\item Find the area of the smaller part of a disc of radius $10 cm$, cut off by a chord $AB$ which subtends an angle of $22\frac{1}{2} \degree$ at the circumference.
     \hfill {(1980)}
 \item Let $ABC$ be the triangle with $AB = 1, AC = 3$ and $\angle BAC=\frac{\pi}{2}$. If a circle of radius $r > 0$ touches the sides $AB, AC$ and also touches internally the circumcircle of the triangle $ABC$, then the value of $r$ is\rule{1cm}{0.1pt}.
	\hfill (2022)
\item Let $G$ be a circle of radius $R > 0$. Let $G_1, G_2, \dots, G_n$ be $n$ circles of equal radius $r > 0$. Suppose each of the $n$ circles $G_1, G_2, \dots, G_n$ touches the circle $G$ externally. Also, for $i = 1, 2, \dots, n - 1$, the circle $G_i$ touches $G_{i+1}$ externally, and $G_n$ touches $G_1$ externally. Then, which of the following statements is/are TRUE?
\hfill (2022)     
\begin{multicols}{2}     \begin{enumerate}
         \item If $n = 4$, then $(\sqrt{2} - 1)r < R$.
         \item If $n = 5$, then $r < R$.
         \item If $n = 8$, then $(\sqrt{2} - 1) r < R$.
         \item If $n = 12$, then $\sqrt{2} (\sqrt{3} + 1) r > R$.
    \end{enumerate} \end{multicols}
\item Consider an obtuse angled triangle $ABC$ in which the difference between the largest and the smallest angle is $\frac{\pi}{2}$ and whose sides are in arithmetic progression. Suppose that the vertices of this triangle lie on a circle of radius 1.
\hfill (2023)
\begin{enumerate}
\item Let $a$ be the area of the triangle $ABC$. Then the value of $(64a)^2$ is
\rule{1cm}{0.1pt}.
\item The in radius of the triangle $ABC$ is
\rule{1cm}{0.1pt}.
\end{enumerate}
%
\item Let $A_1, A_2, A_3, \dots, A_8$ be the vertices of a regular octagon that lie on a circle of radius 2. Let $P$ be a point on the circle, and let $PA_i$ denote the distance between the points $P$ and $A_i$ for $i = 1,2, \dots, 8$. If $P$ varies over the circle, then the maximum value of the product $PA_1 \cdot PA_2 \cdot PA_3 \cdots PA_8$ is
\rule{1cm}{0.1pt}.
\hfill (2023)



\end{enumerate}
