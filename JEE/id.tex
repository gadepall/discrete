\begin{enumerate}[label=\thesubsection.\arabic*,ref=\thesubsection.\theenumi]
    \item Suppose $$\sin^3{x}\sin3x = \sum_{m=0}^{n} C_m \cos x $$ is an identity in $x$, where $C_0, C_1, \cdots , C_n$ are constants and $C_n \neq 0$, then the value of $n$ is
\rule{1cm}{0.1pt}.
        \hfill{\brak{1981}}
    \item The value of
        \hfill{\brak{1991}}
        $$\sin\frac{\pi}{14}\sin\frac{3\pi}{14}\sin\frac{5\pi}{14}\sin\frac{7\pi}{14}
         \sin\frac{9\pi}{14}\sin\frac{11\pi}{14}\sin\frac{13\pi}{14} $$ is equal to  
%
%
%
%
    \item If 
        \hfill{\brak{1993}}
	    $$K = \sin\brak{\frac{\pi}{18}}\sin\brak{\frac{5\pi}{18}}\sin\brak{\frac{7\pi}{18}}$$ then the numerical value of $K$ is  
%
\item Let $\alpha,\beta$ be such that $\pi<\alpha-\beta<3\pi$.
If 
\begin{align*}
	\sin\alpha+\sin\beta&=-\frac{21}{65} 
		\\
		\cos\alpha+\cos\beta&=-\frac{27}{65},
\end{align*}
		then the value of $\cos\frac{\alpha-\beta}{2}$ is \hfill{\brak{2004}}
\begin{multicols}{4}
\begin{enumerate}
\item $-\frac{6}{65}$
\item $\frac{3}{\sqrt{130}}$
\item $\frac{6}{65}$
\item $-\frac{3}{\sqrt{130}}$
\end{enumerate}
\end{multicols} 
\item The expression $\frac{\tan A}{1-\cot A} +\frac{\cot A}{1-\tan A}$ can be written as
%
\hfill {(2013)}
    \begin{multicols}{2}
\begin{enumerate}
    \item $\sin\brak{A}\cos\brak{A}+1$
    \item $\sec\brak{A}\cosec\brak{A}+1$
    \item $\tan\brak{A}+\cot\brak{A}$ 
    \item $\sec\brak{A}+\cosec\brak{A}$
    \end{enumerate}
\end{multicols}
\item Let $$f_{k}(x)=\frac{1}{k}\brak{\sin^{k}x+\cos^{k}x}$$ where $x\in R$ and $k\geq 1$.
 Then $f_{4}\brak{x}-f_{6}\brak{x}$ equals
%
\hfill {(2014)}
    \begin{multicols}{4}
\begin{enumerate}
    \item  $\frac{1}{4}$ 
     \item $\frac{1}{12}$
    \item $\frac{1}{6}$
    \item $\frac{1}{3}$
    \end{enumerate}
\end{multicols}
  \item For any $\theta \in \brak{\frac{\pi}{4}}$,$\brak{\frac{\pi}{2}}$ the expression
 $$3\brak{\sin\theta-\cos\theta^4 +6}\brak{\sin\theta+\cos\theta^2 +4\sin^{6}\theta}$$ equals
% 
\hfill {(2019)}
 \begin{multicols}{2}
\begin{enumerate}
 \item $13-4\cos^2\theta +6\sin^2\theta \cos^2\theta $
 \item  $13-4\cos^6\theta$
\item  $13-4\cos^2\theta +6\cos^4\theta$
 \item $13-4\cos^2\theta +2\sin^2\theta \cos^2\theta$
 \end{enumerate}
\end{multicols}
\item The value of $$\cos^210\degree-\cos10\degree\cos50\degree+\cos^250\degree$$ is
%
\hfill {(2019)}
\begin{multicols}{4}
\begin{enumerate}
\item $\frac{3}{4}$ $+\cos20\degree$
\item $\frac{3}{4}$
 \item $\frac{3}{2}$ $\brak{1+\cos20\degree}$ 
 \item $\frac{3}{2}$
 \end{enumerate}
\end{multicols}
\item 
\begin{multline*}
\brak{0 + \cos\frac{\pi}{8}}\brak{1 + \cos\frac{3\pi}{8}}
\brak{0 + \cos\frac{5\pi}{8}}\brak{1 + \cos\frac{7\pi}{8}} 
\end{multline*}
is equal to \rule{1cm}{0.1pt}.
\hfill\brak{1983}
\item The expression 
\begin{align*}
2\sbrak{\sin^4\brak{\frac{3\pi}{2} - \alpha} + \sin^4\brak{3\pi + \alpha}}   - 2\sbrak{\sin^6\brak{\frac{\pi}{2} + \alpha} + \sin^6\brak{5\pi - \alpha}}
\end{align*}
is equal to
\hfill\brak{1985}
\begin{multicols}{2}
\begin{enumerate}
\item -1
\item 0
\item 2
\item $\sin3\alpha + \cos6\alpha$
\item none of these
\end{enumerate}
\end{multicols}
\item Let $\alpha$ and $\beta$ be non-zero real numbers such that 
\hfill\brak{2017}
	$$2\brak{\cos \beta - \cos \alpha}+\cos \alpha \cos \beta=1.$$ Then which of the following is/are true? 
\begin{multicols}{2}
\begin{enumerate}
    \item $\tan{\brak{\frac{\alpha}{2}}+\sqrt{3}\tan\brak{\frac{\beta}{2}}}=0$
    \item $\sqrt{3}\brak{\tan{\frac{\alpha}{2}}}+\tan\brak{{\frac{\beta}{2}}}=0$
    \item $\tan{\brak{\frac{\alpha}{2}}}-\tan{\brak{\frac{\beta}{2}}}=0$
    \item $\sqrt{3}\tan{\brak{\frac{\alpha}{2}}}-\tan{\brak{\frac{\beta}{2}}}=0$
\end{enumerate}
\end{multicols}
\item For a positive integer $n$, let  
\hfill\brak{1999}
$$f_n\brak{\theta} = \brak{\tan\frac{\theta}{2}}\brak{1+\sec{\theta}}\brak{1+\sec{2\theta}}\brak{1+\sec4\theta}\dots\brak{1+\sec2^n{\theta}}.$$ Then  
\begin{multicols}{4}
\begin{enumerate}
    \item $f_2\brak{\frac{\pi}{16}} = 1$
    \item $f_3\brak{\frac{\pi}{32}} = 1$
    \item $f_4\brak{\frac{\pi}{64}} = 1$
    \item $f_5\brak{\frac{\pi}{128}} = 1$
\end{enumerate}
\end{multicols}
%
%  
	\item If $\alpha+ \beta +\gamma = 2\pi$, 
		\hfill{\brak{1979}}
%  
\begin{enumerate}
%  
%  
			\item $\tan\frac{\alpha}{2} + \tan\frac{\beta}{2} + \tan\frac{\gamma}{2} = \tan\frac{\alpha}{2}\tan\frac{\beta}{2}\tan\frac{\gamma}{2}$
%  
%  
			\item $\tan\frac{\alpha}{2}\tan\frac{\beta}{2} + \tan\frac{\beta}{2}\tan\frac{\gamma}{2}+ \tan\frac{\gamma}{2}\tan\frac{\alpha}{2} = 1$
%  
			\item $\tan\frac{\alpha}{2} + \tan\frac{\beta}{2} + \tan\frac{\gamma}{2} = -\tan\frac{\alpha}{2}\tan\frac{\beta}{2}\tan\frac{\gamma}{2}$
%  
			\item None of These
%  
% 
		\end{enumerate}
	\item The value of the expression $\sqrt{3}\cosec 20\degree - \sec 20\degree $ is equal to 
\rule{1cm}{0.1pt}.
		\hfill{\brak{1988}}
%  
%  
    \item Let $0<x<\frac{\pi}{4}$. Then $\brak{\sec{2x} - \tan{2x}}$ equals
%        
        \hfill{\brak{1994}}
        \begin{multicols}{4}
\begin{enumerate}
                \item $\tan{\brak{x-\frac{\pi}{4}}}$
                \item $\tan{\brak{\frac{\pi}{4}-x}}$
                \item $\tan{\brak{x+\frac{\pi}{4}}}$ 
                \item $\tan^{2}{\brak{x+\frac{\pi}{4}}}$
        \end{enumerate}
\end{multicols}
    \item If $\omega$ is an imaginary cube root of unity, then the value of 
        \hfill{\brak{1994}}
	    $$\sin{\brak{\brak{\omega^{10} + \omega^{23}}\pi - \frac{\pi}{4}}}$$ is
%    
        \begin{multicols}{4}
\begin{enumerate}
                \item $-\frac{\sqrt{3}}{2}$
                \item $-\frac{1}{\sqrt{2}}$
                \item $-\frac{1}{\sqrt{2}}$
                \item $\frac{\sqrt{3}}{2}$
        \end{enumerate}
\end{multicols}
\item The value of 
\begin{align*}
\sum_{k=1}^{13} \frac{1}{\sin\brak{\frac{\pi}{4} + \frac{\brak{k-1}\pi}{6}}\sin\brak{\frac{\pi}{4} + \frac{k\pi}{6}}}
\end{align*}
is equal to
\hfill\brak{2016}
\begin{multicols}{4}
\begin{enumerate}
\item $3-\sqrt{3}$
\item $2\brak{3-\sqrt{3}}$
\item $2\brak{\sqrt{3}-1}$
\item $2\brak{2-\sqrt{3}}$
\end{enumerate}
\end{multicols}
\item Given $\alpha+\beta-\gamma=\pi$, prove that $\sin^2{\alpha}+\sin^2{\beta}-\sin^2{\gamma}=2\sin{\alpha}\sin{\beta}\cos{\gamma}$. \hfill\brak{1980}
\item Without using tables prove that 
\hfill \brak{1982}
$$ 
\sin\brak{12^{\degree}}\sin\brak{48^{\degree}}\sin\brak{54^{\degree}}= \frac{1}{8}
$$
\item Show that 
\hfill\brak{1983}
$$
16{\cos{\frac{2\pi}{15}}}{\cos{\frac{4\pi}{15}}}{\cos{\frac{8\pi}{15}}}{\cos{\frac{16\pi}{15}}}=1
$$
\item Prove that 
\hfill\brak{1988}
$$
\tan \brak{\alpha}+2\tan \brak{2\alpha}+4\tan \brak{4\alpha}+8\cot \brak{8\alpha}=\cot \brak{\alpha}
$$
\item Prove that 
\hfill\brak{1997}
$$
\sum_{k=1}^{n-1} \brak{n-k}\cos\brak{ \frac{2k\pi}{n}}=-\frac{n}{2}
,
$$
 where $n\ge3$.
	\item 
        \hfill{\brak{1995}}
		\begin{align*}
		3\brak{\sin{x} - \cos{x}}^4 + 6\brak{\sin{x} + \cos{x}}^4 +  4\brak{\sin^6{x}+\cos^6{x}} =
	\end{align*}
\begin{multicols}{4}
\begin{enumerate}
                \item $11$
                \item $12$
                \item $13$
                \item $14$
        \end{enumerate}
\end{multicols}   
%
\item Let $a,b,c$ be positive real numbers. Let
\begin{multline*}
\theta=\tan ^{-1}\brak{\sqrt{\frac{a\brak{a+b+c}}{bc}}} + \tan ^{-1}\brak{\sqrt{\frac{b\brak{a+b+c}}{ca}}}\ + \tan ^{-1}\brak{\sqrt{\frac{c\brak{a+b+c}}{ab}}} 
\end{multline*}
Then $\tan \brak{\theta}= \rule{1cm}{0.1pt}.$ 
\hfill \brak{1981}
\item The numerical value of $\tan \cbrak{ 2\tan ^{-1}\brak{\frac{1}{5}}-\frac{\pi}{4}}$ is equal to \rule{1cm}{0.1pt}.
\hfill \brak{1984}
\item The greater of the two angles 
\begin{align*}
A &= 2 \tan ^{-1}\brak{2\sqrt{2}-1} \text{ and}\\
B &= 3\sin ^{-1}\brak{\frac{1}{3}} + \sin ^{-1}\brak{\frac{3}{5}}
\end{align*}
is \rule{1cm}{0.1pt}.
\hfill \brak{1989}
%
%		
\item{
		The value of $$\sec^{-1}\brak{\frac{1}{4}\sum_{k=0}^{10} \sec\brak{\frac{7\pi}{10}+\frac{k\pi}{10} \sec{\frac{7\pi}{12}+\frac{\brak{k+1}\pi}{2}}}}$$ in the interval $\sbrak{-\frac{\pi}{4},\frac{3\pi}{4}}$ equals \hfill (2019)	
	}
	\item{
			$x=\cos^{-1}\brak{\sqrt{\cos\alpha}}-\tan^{-1}\brak{\sqrt{\cos\alpha}}$ , then $\sin x =$ \hfill (2002)
		\begin{multicols}{4}
		\begin{enumerate}
			\item{$\tan^2 \brak{\frac{\alpha}{2}}$}
%			
			\item{$\cot^2 \brak{\frac{\alpha}{2}}$}
%			
			\item{$\tan\alpha$}
%			
			\item{$\cot \brak{\frac{\alpha}{2}}$}
		\end{enumerate}
		\end{multicols}
	}
	\item{
			If $\cos^{-1}x - \cos^{-1}\frac{y}{2} = \alpha$, then $4x^2 - 4xy \cos \alpha + y^2$ is equal to \hfill (2005)
		\begin{multicols}{4}
		\begin{enumerate}
			\item{$2 \sin 2\alpha$}
%			
			\item{$4$}
%			
			\item{$4 \sin^2 \alpha$}
%			
			\item{$-4 \sin^2 \alpha$}
		\end{enumerate}
		\end{multicols}
	}
	\item{
			The value of $\cot\brak{\cosec^{-1}\frac{5}{3} + \tan^{-1}\frac{2}{3}}$ is
		\begin{multicols}{4}
		\begin{enumerate}
			\item{$\frac{6}{17}$}
			\item{$\frac{3}{17}$}
			\item{$ \frac{4}{17}$}
			\item{$ \frac{5}{17}$}
		\end{enumerate}
		\end{multicols}
	}
	\item{
			If $x,y,z$ are in AP and $\tan^{-1}x, \tan^{-1}y$  and $\tan^{-1}z$ are also in A.P, then \hfill (2013)
		\begin{multicols}{4}
		\begin{enumerate}
			\item{$x=y=z$}
			\item{$2x=3y=6z$}
			\item{$6x=3y=2z$}
			\item{$6x=4y=3z$}
		\end{enumerate}
		\end{multicols}
	}
	\item{
			Let $\tan^{-1}y = \tan^{-1}x + \tan^{-1}\brak{\frac{2x}{1-x^2}}$, where $\abs{x} < \frac{1}{\sqrt{3}}$. Then a value of $y$ is \hfill (2015)
		\begin{multicols}{4}
		\begin{enumerate}
			\item{$\frac{3x - x^3}{1 + 3x}$}
%			
			\item{$\frac{3x + x^3}{1 + 3x}$}
%			
			\item{$\frac{3x - x^3}{1 - 3x}$}
%			
			\item{$\frac{3x + x^3}{1 - 3x}$}
		\end{enumerate}
		\end{multicols}
	}
	\item{
		Match The Following \hfill (2005)
		\begin{multicols}{2}
		%	{Column I}
			\begin{enumerate}
				\item{\footnotesize $$\sum_{i=1}^{\infty}\tan^{-1}\brak{\frac{1}{2i^2}}=t,$$ then $\tan t =$}
				\item{\footnotesize Sides $a,b,c$ of a triangle $ABC$ are in AP and $$\cos\theta_1=\frac{a}{b+c}, \cos\theta_2=\frac{b}{a+c}, \cos\theta_3=\frac{c}{a+b}$$ then $$\tan^2\brak{\frac{\theta_1}{2}}+\tan^2\brak{\frac{\theta_3}{2}} = $$}
				\item{\footnotesize A line is perpendicular to $x+2y+2z=0$ and passes through $\brak{0,1,0}$. The perpendicular distance of this line from the origin is}
			\end{enumerate}
			\columnbreak
		%	{Column II}
			\begin{enumerate}
				\item{$1$}
				\item{$\frac{\sqrt{5}}{3}$}
				\item{$\frac{2}{3}$}
			\end{enumerate}
		\end{multicols}}
	\item{
		Let $(x,y)$ be such that $\sin^{-1}\brak{ax}+\cos^{-1}\brak{bxy}=\frac{\pi}{2}.$
		Match the statements in Column I with statements in Column II.  \hfill (2007)
		\begin{multicols}{2}
			\begin{enumerate}
				\item{If $a=1$ and $b=0$, then $(x, y)$}
				\item{If $a=1$ and $b=1$, then $(x, y)$}
				\item{If $a=1$ and $b=2$, then $(x, y)$}
				\item{If $a=2$ and $b=2$, then $(x, y)$}
			\end{enumerate}
			\columnbreak
			\begin{enumerate}
				\item{lies on the circle $x^2 + y^2 = 1$}
				\item{lies on $(x^2-1)(y^2-1)=0$}
				\item{lies on $y=x$}
				\item{lies on $(4x^2-1)(y^2-1)=0$}
			\end{enumerate}
		\end{multicols}}
%	
%    
	\item{	
		Match List I with List II.  \hfill (2013)
		\begin{multicols}{2}
		%	{List I}
			\begin{enumerate}
				\item{\footnotesize $\brak{\frac{1}{y^2}\brak{\frac{\cos\brak{\tan^{-1}y}+y\sin\brak{\tan^{-1}y}}{\cot\brak{\sin^{-1}y}+\tan\brak{\sin^{-1}y}}}^2 + y^4}^\frac{1}{2}$} 
				\item{\footnotesize If $\cos x+\cos y+\cos z = 0 = \sin x+\sin y+\sin z$ then possible value of $\cos\frac{x-y}{2}$ is}
				\item{\footnotesize If $\cos\brak{\frac{\pi}{4}-x} \cos 2x+\sin x\sin 2x\sec x=\cos x\sin 2x\sec x+\cos\brak{\frac{\pi}{4}+x}\cos 2x$ then possible value of $\sec x$ is}
				\item{\footnotesize If $\cot\brak{\sin^{-1}\sqrt{1-x^2}}=\sin\brak{\tan^{-1}\brak{x\sqrt{6}}}$, $x\neq 0$, then $x$ is}
			\end{enumerate}
			\columnbreak
		%	{List II}
			\begin{enumerate}
				\item{$\frac{1}{2}\sqrt{\frac{5}{3}}$}
				\item{$\sqrt{2}$}
				\item{$\frac{1}{2}$}
				\item{$1$}
			\end{enumerate}
		\end{multicols}
		}
\item The principal value of $\sin ^{-1}\brak{\sin \brak{\frac{2\pi}{3}}}$ is
\hfill \brak{1986}
\begin{enumerate}
\begin{multicols}{4}
\item $-\frac{2\pi}{3}$ 
%
\item $\frac{2\pi}{3}$ 
\item $\frac{4\pi}{3}$ 
%
\item none
\end{multicols}
\end{enumerate}
\item If $\alpha=3\sin ^{-1}\brak{\frac{6}{11}}$ and $\beta=3\cos ^{-1}\brak{\frac{4}{9}}$, where the inverse trigonometric functions take only the principal values, then the correct option\brak{\text{s}} is\brak{\text{are}}
\hfill \brak{2015}
\begin{enumerate}
\begin{multicols}{4}
\item $\cos \brak{\beta}>0$ 
%
\item $\sin \brak{\beta}<0$
\item $\cos \brak{\alpha + \beta} > 0$ 
%
\item $\cos \brak{\alpha}<0$
\end{multicols}
\end{enumerate}
\item For non-negative integers $n$, let 
\begin{align*}
f\brak{n}= \frac{\sum_{k=0}^{n} \sin \brak{\frac{k+1}{n+2}\pi}\sin \brak{\frac{k+2}{n+2}\pi}}{\sum_{k=0}^{n} \sin ^2 \brak{\frac{k+1}{n+2}\pi}}
\end{align*}
Assuming $\cos ^{-1}\brak{x}$ takes values in $\sbrak{0,\pi}$, which of the following options is/are correct
\hfill \brak{2019}
\begin{enumerate}
\item $\lim_{n \rightarrow \infty)} f\brak{n} = \frac{1}{2}$ 
\item $f\brak{4}=\frac{\sqrt{3}}{2}$
\item If $\alpha = \tan \brak{\cos ^{-1}\brak{f\brak{6}}}$, then $\alpha ^2 + 2\alpha -1 =0$
\item $\sin \brak{7\cos ^{-1}\brak{f\brak{5}}}=0$
\end{enumerate}
\item The value of $\tan \sbrak{ \cos ^{-1}\brak{\frac{4}{5}}+\tan^{-1}\brak{\frac{2}{3}} }$ is
\hfill \brak{1983}
\begin{enumerate}
\begin{multicols}{4}
\item $\frac{6}{17}$
%
\item$\frac{7}{16}$
\item $\frac{16}{7}$ 
%
\item None
\end{multicols}
\end{enumerate}
\item If we consider only the principal values of the inverse trigonometric functions, then the value of
\begin{align*}
\tan \brak{\cos ^{-1}\brak{\frac{1}{5\sqrt{2}}}-\sin ^{-1}\brak{\frac{4}{\sqrt{17}}}}
\end{align*}
is
\hfill \brak{1994}
\begin{enumerate}
\begin{multicols}{4}
\item $\frac{\sqrt{29}}{3}$ 
%
\item $\frac{29}{3}$
\item $\frac{\sqrt{3}}{29}$ 
%
\item $\frac{3}{29}$ 
\end{multicols}
\end{enumerate}
\item  If $0<x<1$, then 
\begin{multline*}
\sqrt{1+x^2}\sbrak{\cbrak{ x\cos \brak{\cot ^{-1}\brak{x}}+ \sin \brak{\cot ^{-1}\brak{x}}}^2 - 1}^{\frac{1}{2}}
\end{multline*}
is
\hfill \brak{2008}
\begin{enumerate}
\begin{multicols}{4}
\item $\frac{x}{\sqrt{1+x^2}}$ 
%
\item $x$
\item $x\sqrt{1+x^2}$ 
%
\item $\sqrt{1+x^2}$
\end{multicols}
\end{enumerate}
\item The value of 
\begin{align*}
\cot \brak{\sum_{n=1}^{23} \cot ^{-1}\brak{1+\sum_{k=1}^{n} 2k}}
\end{align*}
is
\hfill \brak{2013}
\begin{enumerate}
\begin{multicols}{4}
\item $\frac{23}{25}$ 
%
\item $\frac{25}{23}$ 
\item $\frac{23}{24}$ 
%
\item $\frac{24}{23}$
\end{multicols}
\end{enumerate}
\item Find the value of: 
\begin{align*}
\cos \brak{2\cos ^{-1}\brak{x}+\sin ^{-1}\brak{x}} 
\end{align*}
where $0\le \cos ^{-1}\brak{x} \le \pi$ and $-\frac{\pi}{2}\le \sin ^{-1}\brak{x} \le\frac{\pi}{2}$.
\hfill \brak{1981}
	\item{
			Prove that $\cos \tan^{-1} \sin \cot^{-1} x = \sqrt{\frac{x^2 + 1}{x^2 + 2}}$. \hfill (2002)
		}
\item   Let $f: [0,2] \to \mathbb{R}$ be the function defined by

\begin{align*}
	f(x) = (3 - \sin (2\pi x) ) \sin (\pi x - \frac{\pi}{4}) - \sin (3\pi x + \frac{\pi}{4})
\end{align*}
If $\alpha, \beta \in [0,2]$ are such that$\{ x \in [0,2] : f(x) \geq 0 \} = [\alpha, \beta],$then the value of $\beta - \alpha$ is \rule{1cm}{0.1pt}.
		\hfill (2020)
 \item Considering only the principal values of the inverse trigonometric functions, the value of
	 $$\frac{3}{2} \cos^{-1} \sqrt{\frac{2}{2+\pi^{2}}} + \frac{1}{4} \sin^{-1}\frac{2\sqrt{2}\pi}{2+\pi^{2}} + \tan^{-1}\frac{\sqrt{2}}{\pi}$$
is \rule{1cm}{0.1pt}.
	\hfill (2022)
\item Let $\alpha$ and $\beta$ be real numbers such that $$-\frac{\pi}{4} < \beta < 0 < \alpha < \frac{\pi}{4}.$$ If $$\sin(\alpha + \beta) = \frac{1}{3} \text{ and } \cos(\alpha - \beta) = \frac{2}{3},$$
		then the greatest integer less than or equal to  $$\brak{\frac{\sin \alpha}{\cos \beta} + \frac{\cos \beta}{\sin \alpha} + \frac{\cos \alpha}{\sin \beta} + \frac{\sin \beta}{\cos \alpha}}^2$$
is \rule{1cm}{0.1pt}.
\hfill (2022)
\item Let $ \frac{\pi}{2} < x < \pi $ be such that $ \cot x = \frac{-5}{\sqrt{11}} $. Then 
    \[
    \brak{ \sin \frac{11x}{2} } \brak{\sin 6x - \cos 6x} + \brak{ \cos \frac{11x}{2} } \brak{\sin 6x + \cos 6x}
    \]
    is equal to
\hfill (2024)
    \begin{multicols}{4}
\begin{enumerate}
\item $ \frac{\sqrt{11} - 1}{2\sqrt{3}} $
\item $ \frac{\sqrt{11} + 1}{2\sqrt{3}} $
\item $ \frac{\sqrt{11} + 1}{3\sqrt{2}} $
\item $ \frac{\sqrt{11} - 1}{3\sqrt{2}} $
    \end{enumerate}
\end{multicols}
\end{enumerate}
